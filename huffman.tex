\documentclass[a4paper, 12pt]{article}
\usepackage[utf8]{inputenc}
\usepackage[T1]{fontenc}
\usepackage[french]{babel}
\usepackage{amsmath}
\usepackage{amssymb}

\let\iff\Longleftrightarrow

\pagestyle{headings}

\title{Huffman}
\author{Quentin Januel, Loïc Mohin et Anthony Villeneuve}
\date{\today}

\begin{document}

\maketitle
\newpage

\tableofcontents{}
\newpage

\section{Définitions}

\subsection{Alphabet}
On appelle $\Sigma$ un alphabet dont les éléments sont appelés des lettres. \\
Un mot sur $\Sigma$ est un $n$-uplet de lettres : $m = (a_1,\ a_2,\ ...,\ a_n)$. \\
L'ensemble des mots sur $\Sigma$ est noté $\Sigma^* := \{m \in \Sigma^n,\ \forall n \in \mathbb{N}\}$. \\
Soit $m = (a_1,\ a_2,\ ...,\ a_n)$ un mot, on appelle longueur du mot $m$ notée $|m|$ l'entier $n$. \\
Enfin, on note $\varepsilon$ le mot vide (unique mot de longueur $0$). \\ \\
On peut munir $\Sigma^*$ d'une loi de composition interne, la concaténation $+$ : \\
$(a_1,\ ...,\ a_n)+(b_1,\ ...,\ a_n) = (a_1,\ ...,\ a_n,\ b_1,\ ...,\ a_n)$. \\
On observe alors que $(\Sigma^*,\ +)$ est un monoïde.

\subsection{Arbre binaire}
Soit $E$ un ensemble, on dit que $A := (Q,\ T)$ est un arbre binaire sur $E$ avec $Q \subset E$ et $T \subset E\times E$ s'il respecte les 3 propriétés suivantes :
\begin{enumerate}
\item $\exists \ !\ r \in Q,\ \forall x \in Q,\ (x, r) \notin T$ ($r$ est appelé racine de $A$ noté $r(A)$),
\item $\forall x_2 \in Q\backslash\{r\},\ \exists \ !\ x_1 \in Q,\ (x_1,\ x_2) \in T$,
\item $\forall x_1 \in Q,\ \text{card}(\{x_2 \in Q,\ (x_1,\ x_2)\in T\}) \leq 2$.
\end{enumerate}
Les éléments de $Q$ sont appelés les états et les éléments de $T$ transitions.

\subsection{Arbre de Huffman}
On appelle arbre de Huffman un arbre binaire $A$ sur $\mathbb{N}$ qui respecte les propriétés suivantes :
\begin{itemize}
\item $n(A) = n(g(A))+n(d(A))$,
\item $g(A)$ et $d(A)$ sont tous deux des arbres de Huffman,
\item ... (il manque un truc ici, à faire !).
\end{itemize}
On note $\mathcal{H}$ l'ensemble des arbres de Huffman et on a $\mathcal{H} \subset \mathcal{A}_\mathbb{N}$.

\section{Classes d'équivalence d'arbres de Huffman}
Intuitivement, on aimerait définir des classes d'équivalences des arbres de Huffman selon la longueur de la chaine de bits qu'ils encodent. Tâchons donc d'abord de pouvoir évaluer cette valeur à l'aide d'une fonction :
$$
\begin{matrix}
\omega: &\mathcal{H} &\rightarrow &\mathbb{N} \\
&A &\mapsto &{\begin{cases}
	0 \text{ si } A = A_0 \\    
	n(A)+\omega(g(A))+\omega(d(A)) \text{ sinon}
\end{cases}}
\end{matrix} \\
$$
Nous pouvons à présent définir $\mathcal{R}$ notre relation d'équivalence. Soient $A, B \in \mathcal{H}$, on a alors
$$
A \mathcal{R} B \iff \omega(A) = \omega(B)
$$
On dénote également $\overline{\mathcal{H}} := \mathcal{H}/\mathcal{R}$ l'ensemble quotient de $\mathcal{H}$ par $\mathcal{R}$.

\end{document}
