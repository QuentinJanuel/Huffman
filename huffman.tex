\documentclass[a4paper, 12pt]{article}
\usepackage[utf8]{inputenc}
\usepackage[T1]{fontenc}
\usepackage[french]{babel}
\usepackage{amsmath}
\usepackage{amssymb}
\usepackage{relsize}
\let\iff\Longleftrightarrow
\newtheorem{theo}{Théorème}
\pagestyle{headings}
\title{Huffman}
\author{
\textbf{Auteurs :} Quentin Januel, Loïc Mohin et Anthony Villeneuve \\
\textbf{Mentors :} Olivier Gipouloux et Stéphane Gaussent \\ \\
\textbf{Fait à :} Université Jean Monnet, Saint Etienne
}
\date{\today}
\begin{document}
\maketitle
\newpage
\tableofcontents{}
\newpage

\section{Prérequis}

Dans cette section, nous allons tâcher de définir les outils dont nous aurons besoin pour l'analyse des arbres de Huffman.

\subsection{Alphabet}
On appelle $\Sigma$ un alphabet dont les éléments sont appelés des lettres. \\
Un mot sur $\Sigma$ est un $n$-uplet de lettres : $m = (a_1,\ a_2,\ ...,\ a_n)$. \\
L'ensemble des mots sur $\Sigma$ est noté $\Sigma^* := \{m \in \Sigma^n,\ \forall n \in \mathbb{N}\}$. \\
Soit $m = (a_1,\ a_2,\ ...,\ a_n)$ un mot, on appelle longueur du mot $m$ notée $|m|$ l'entier $n$. \\
Enfin, on note $\varepsilon$ le mot vide (unique mot de longueur $0$). \\ \\
On peut munir $\Sigma^*$ d'une loi de composition interne, la concaténation $+$ : \\
$(a_1,\ ...,\ a_n)+(b_1,\ ...,\ b_n) = (a_1,\ ...,\ a_n,\ b_1,\ ...,\ b_n)$. \\
On observe alors que $(\Sigma^*,\ +)$ est un monoïde.

\subsection{Mot pondéré}
On dit qu'un élement $(a,\ n)\in \Sigma^*\times \mathbb{N}$ est un mot pondéré de $\Sigma$ et on notera :
\begin{enumerate}
\item $l((a,\ n)) = a$ ($l$ pour "lettre"),
\item $p((a,\ n)) = n$ ($p$ pour "poids")
\end{enumerate}
On définit alors la somme de mots pondérés $x+y = (l(x)+l(y),\ p(x)+p(y))$ et on se retrouve avec un nouveau monoïde :  $(\Sigma^*\times \mathbb{N},\ +)$.

\subsection{Arbre binaire}
Soit $E$ un ensemble, on dit que $A := (Q,\ T)$ est un arbre binaire sur $E$ avec $Q \subset E$ et $T \subset E\times \mathbb{F}_2\times E$ s'il respecte les 3 propriétés suivantes :
\begin{enumerate}
\item $\exists \ !\ r \in Q,\ \forall (x,\ b) \in Q\times \mathbb{F}_2,\ (x, b, r) \notin T$ ($r$ est appelée racine de $A$ notée $r(A)$),
\item $\forall x_2 \in Q\backslash\{r\},\ \exists \ !\ (x_1,\ b) \in Q\times \mathbb{F}_2,\ (x_1,\ b,\ x_2) \in T$, % (on dit que $x_1$ est un parent de $x_2$ que l'on note $p(x_2)$),
\item $\forall (x_1,\ b) \in Q\times \mathbb{F}_2,\ \text{card}(\{x_2 \in Q,\ (x_1,\ b,\ x_2)\in T\}) \leq 1$.
\end{enumerate}
Les éléments de $Q$ (notés $q(A)$) sont appelés les états et les éléments de $T$ (notés $t(A)$) transitions. \\
Enfin, l'ensemble des arbres binaires sur $E$ est noté $\mathcal{A}_E$.
\newpage

\section{Arbre de Huffman}

\subsection{Définition}
Prennons un alphabet $\Sigma$ quelconque. Tout arbre de la forme
$$
(\{x\},\ \emptyset),\ x \in \Sigma^*\times \mathbb{N},\ |x| = 1
$$
est appelé arbre de Huffman sur $\Sigma$. \\
De plus, soient $A$ et $B$ deux arbres de Huffman et $r := r(A)+r(B)$, alors
$$
M_{A,\ B} := (q(A) \cup q(B) \cup \{r\},\ t(A)\cup(B)\cup \{(r,\ 0,\ r(A)),\ (r,\ 1,\ r(B))\})
$$
est également un arbre de Huffman, on dit que $M$ est la fusion de $A$ et de $
B$. \\
Notons $\mathcal{H}_\Sigma$ l'ensemble des arbres de Huffman sur $\Sigma$. \\
On pose aussi
$$
\begin{matrix}
m: &\mathcal{H}_\Sigma\times \mathcal{H}_\Sigma &\rightarrow &\mathcal{H}_\Sigma \\
&(A,\ B) &\mapsto &M_{A,\ B}
\end{matrix}
$$

\subsection{Théorème}
Pour tout alphabet $\Sigma$, on a $\mathcal{H}_\Sigma \subset \mathcal{A}_{\Sigma^*\times \mathbb{N}}$. \\
\textbf{Preuve :} \\
A faire (facile, montrer que $M_{A,\ B}$ respecte les 3 propriétés d'un arbre binaire) \\

\subsection{Classification des arbres de Huffman}
Blabla
$$
\begin{matrix}
\omega: &\mathcal{H}_\Sigma &\rightarrow &\mathbb{N} \\
&A &\mapsto &\mathlarger{\mathlarger{‎‎\sum}}_{(x_1,\ b,\ x_2)\in t(A)} p(x_2)‎‎
\end{matrix} \\
$$
On définit la relation d'équivalence
$$
A\mathcal{R}B \iff \omega(A) = \omega(B),\ \forall A, B \in \mathcal{H}_\Sigma
$$
On dénote également $\overline{\mathcal{H}_\Sigma} := \mathcal{H}_\Sigma/\mathcal{R}$ l'ensemble quotient de $\mathcal{H}_\Sigma$ par $\mathcal{R}$.

\subsection{Théorème}
$\forall A,\ B \in \mathcal{H}_\Sigma,\ \omega\circ m(A,\ B) = \omega(A)+p\circ r(A)+\omega(B)+p\circ r(B)$ \\
\textbf{Preuve :} \\
A faire

% \subsection{Fonctions gauche et droite d'un arbre binaire}
% Soit $E$ un ensemble, pour tout $b \in \mathbb{F}_2$, on pose
% $$
% S_b := \{A \in \mathcal{A}_E,\ \exists \ x \in q(A),\ (r(A), b, x) \in t(A)\}
% $$
% Cela correspond à l'ensemble des arbres binaires qui ont un sous arbre ou gauche ou droit selon la valeur de $b$. \\
% Ensuite, $\forall A \in S_b$, soit $s_b(A)$ l'unique $x \in q(A)$ tel que $(r(A),\ b,\ x) \in t(A)$. \\
% Finalement, on peut poser
% $$
% \begin{matrix}
% \theta_b: &S_b &\rightarrow &\mathcal{A}_E \\
% &A &\mapsto &A_b
% \end{matrix}
% $$
% avec $A_b$ l'unique arbre binaire dont l'ensemble des transitions est aussi grand que possible tel que
% \begin{gather}
% r(A_b) = s_b(A) \\
% t(A_b) \subset t(A)
% \end{gather}
% Utilisons cette fonction pour définir $g := \theta_0$ et $d := \theta_1$. \\
% Pour synthétiser, $g$ et $d$ sont deux fonctions qui rendent le sous arbre binaire respectivement gauche ou droite, définies uniquement si ce sous arbre existe. \\

% \subsection{Arbre de Huffman}
% Soit $(A_i)_{1 \leq i \leq n} \in \mathcal{A}_\mathbb{N}$ un $n$-uplet d'arbres binaires sur \mathbb{N}. \\
% On définit $h$ une fonction qui à $A_i$ associe 
% On appelle arbre de Huffman un arbre binaire $A$ sur $\mathbb{N}$ qui respecte les propriétés suivantes :
% \begin{itemize}
% \item $n(A) = n(g(A))+n(d(A))$,
% \item $g(A)$ et $d(A)$ sont tous deux des arbres de Huffman,
% \item ... (il manque un truc ici, à faire !).
% \end{itemize}
% On note $\mathcal{H}$ l'ensemble des arbres de Huffman et on a $\mathcal{H} \subset \mathcal{A}_\mathbb{N}$.

% \section{Classes d'équivalence d'arbres de Huffman}
% Intuitivement, on aimerait définir des classes d'équivalences des arbres de Huffman selon la longueur de la chaine de bits qu'ils encodent. Tâchons donc d'abord de pouvoir évaluer cette valeur à l'aide d'une fonction :
% $$
% \begin{matrix}
% \omega: &\mathcal{H} &\rightarrow &\mathbb{N} \\
% &A &\mapsto &{\begin{cases}
% 	0 \text{ si } A = A_0 \\    
% 	n(A)+\omega(g(A))+\omega(d(A)) \text{ sinon}
% \end{cases}}
% \end{matrix} \\
% $$
% Nous pouvons à présent définir $\mathcal{R}$ notre relation d'équivalence. Soient $A, B \in \mathcal{H}$, on a alors
% $$
% A \mathcal{R} B \iff \omega(A) = \omega(B)
% $$
% On dénote également $\overline{\mathcal{H}} := \mathcal{H}/\mathcal{R}$ l'ensemble quotient de $\mathcal{H}$ par $\mathcal{R}$.

\end{document}
