\documentclass[a4paper, 12pt]{article}
\usepackage[utf8]{inputenc}
\usepackage[T1]{fontenc}
\usepackage[french]{babel}
\usepackage{amsmath}
\usepackage{amssymb}

\let\iff\Longleftrightarrow

\pagestyle{headings}

\title{Huffman}
\author{Quentin Januel, Loïc Mohin et Anthony Villeneuve}
\date{\today}

\begin{document}

\maketitle
\newpage

\tableofcontents{}
\newpage

\section{Définitions}

\subsection{Alphabet}
On appelle $\Sigma$ un alphabet dont les éléments sont appelés des lettres. \\
Un mot sur $\Sigma$ est un $n$-uplet de lettres : $m = (a_1,\ a_2,\ ...,\ a_n)$. \\
L'ensemble des mots sur $\Sigma$ est noté $\Sigma^* := {m \in \Sigma^n,\ \forall n \in \mathbb{N}}$. \\
Soit $m = (a_1,\ a_2,\ ...,\ a_n)$ un mot, on appelle longueur du mot $m$ notée $|m|$ l'entier $n$. \\
Enfin, on note $\varepsilon$ le mot vide (unique mot de longueur $0$). \\ \\
On peut munir $\Sigma^*$ d'une loi de composition interne, la concaténation $+$ : \\
$(a_1,\ ...,\ a_n)+(b_1,\ ...,\ a_n) = (a_1,\ ...,\ a_n,\ b_1,\ ...,\ a_n)$. \\
On observe alors que $(\Sigma^*, +)$ est un monoïde.

\subsection{Arbre binaire}
Soit $E$ un ensemble, on appelle alors $a$ un arbre binaire sur $E$ un triplet $(G, x, D)$ avec $x$ un élement de $E$ et $G, D$ des arbres binaires. \\
On le définit par récurrence ainsi :
\begin{itemize}
\item L'arbre vide, noté $a_0$, est un arbre binaire,
\item Pour tout G, D arbres binaires, pour tout $x \in E$, $(G,\ x,\ D)$ est un arbre binaire.
\end{itemize}
On dit que $G$ (resp. $D$) est le sous arbre gauche (resp. droit) de $a = (G,\ x,\ D)$ noté $g(a)$ (resp. $d(a)$). $x$ est le noeud de $a$, noté $n(a)$. \\
On note aussi $\mathcal{A}_E$ l'ensemble des arbres binaires sur $E$.

\subsection{Arbre de Huffman}
On appelle arbre de Huffman un arbre binaire $a$ sur $\mathbb{N}$ qui respecte les propriétés suivantes :
\begin{itemize}
\item $n(a) = n(g(a))+n(d(a))$,
\item $g(a)$ et $d(a)$ sont tous deux des arbres de Huffman,
\item ... (il manque un truc ici, à faire !).
\end{itemize}
On note $\mathcal{H}$ l'ensemble des arbres de Huffman et on a $\mathcal{H} \subset \mathcal{A}_\mathbb{N}$.

\section{Classes d'équivalence d'arbres de Huffman}
Intuitivement, on aimerait définir des classes d'équivalences des arbres de Huffman selon la longueur de la chaine de bits qu'ils encodent. Tâchons donc d'abord de pouvoir évaluer cette valeur à l'aide d'une fonction :
$$
\begin{matrix}
\omega: &\mathcal{H} &\rightarrow &\mathbb{N} \\
&a &\mapsto &{\begin{cases}
	0 \text{ si } a = a_0 \\    
	n(a)+\omega(g(a))+\omega(d(a)) \text{ sinon}
\end{cases}}
\end{matrix} \\
$$
Nous pouvons à présent définir $\mathcal{R}$ notre relation d'équivalence. Soient $a, b \in \mathcal{H}$, on a alors
$$
a \mathcal{R} b \iff \omega(a) = \omega(b)
$$
On dénote également $\overline{\mathcal{H}} := \mathcal{H}/\mathcal{R}$ l'ensemble quotient de $\mathcal{H}$ par $\mathcal{R}$.

\end{document}
